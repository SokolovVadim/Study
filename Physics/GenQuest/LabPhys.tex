% "Станет проще"

\documentclass[a4paper,12pt]{article} % тип документа

% report, book

%  Русский язык
% процент означает комментарий

\usepackage[T2A]{fontenc}			% кодировка
\usepackage[utf8]{inputenc}			% кодировка исходного текста
\usepackage[english,russian]{babel}	% локализация и переносы

% Буквенные списки
\renewcommand{\theenumi}{(\asbuk{enumi})}
\renewcommand{\labelenumi}{\asbuk{enumi})}







% Математика
\usepackage{amsmath,amsfonts,amssymb,amsthm,mathtools} 


\usepackage{wasysym}

%Заговолок



\newenvironment{bottompar}{\par\vspace*{\fill}}{\clearpage}

\begin{document} % начало документа


\begin{titlepage}
\newcommand{\HRule}{\rule{\linewidth}{0.5mm}} % Defines a new command for the horizontal lines, change thickness here

\center % Center everything on the page
 
\textsc{\LARGE министерство науки и высшего образования российской федерации\\[0.1cm]
\Large Московский\\[-0.2cm]Физико-Технический Институт\\[0.1cm]\large (государственный университет)}\\[1.5cm] % Name of your university/college
\textsc{\Large Кафедра общей физики}\\[0.2cm] % Major heading such as course name
\textsc{\large Вопрос по выбору, 2 семестр}\\[0.5cm] % Minor heading such as course title

\HRule
\\[0.1cm]
{ \large \bfseries Изучение равновесия 
\\[0.0cm] диссоциации  \(N_2O_4\) в газовой фазе \\[0.1cm] спектрофотометрическим методом} % Title of your document
\HRule
\\[1.5cm]

\begin{flushleft} \large
	\textsf{Студент}\\[0.1cm]
	Соколов Вадим \\
	718 группа
\end{flushleft}

\begin{bottompar}
	\begin{center}
		\includegraphics[width = 80 mm]{logo.jpg}
	\end{center}
	{\large \today}

\end{bottompar}
\vfill
\end{titlepage}


\newpage

\subsection*{Цель работы}
Измерение степение диссоциации молекул \( N_2O_4\).

\subsection*{Оборудование}
Двухлучевой сканирующий спектрофотометр Shimadzu UV-1800; ячейка, наполненная \(N_2O_4\).

\begin{center}
Введение
\end{center}

Положение химического равновесия может быть с высокой точностью рассчитано на основании термодинамических функций участников процесса. Такие расчеты являются основой решения множества важных практических задач. С другой стороны, термодинамические функции реакций чаще всего получают именно из данных по химическим равновесиям, хотя есть и другие экспериментальные источники, например, калориметрия, спектроскопия, теплоемкость и др. Специалистам разных областей химии, биологии, физики и техники необходимо свободно решать как прямые, так и обратные задачи химических равновесий.
В настоящей работе предлагается познакомиться с этими методами на примере равновесия диссоциации

\begin{equation}\label{chem}
N_2O_4 \Leftrightarrow 2NO_2
\end{equation}
в газовой фазе.

Для измерения степени диссоциации в данной работе используется интенсивное поглощение молекулами NO2 света в видимой области спектра. Измерения проводят при разных температурах, получая температурную зависимость константы равновесия \eqref{chem}, которую анализируют в рамках имеющейся теории.

\begin{center}
{ \LARGE теоретические основы}
\end{center}
Условие химического равновесия, как известно, записывается в форме равенства нулю изменения термодинамического потенциала системы в ходе реакции, т. е.
\begin{equation}
 \Delta G_{T,p}  = 0
\end{equation}	

Согласно определению

\begin{equation}
 \Delta G_{T,p}  = \sum {(\frac{\partial G}{\partial n_i})_{T,p} dn_i \sum {\mu_i dn_i}  }
\end{equation}	
где \(n_i\), — число молей i-го компонента системы 1, а \(\mu_i\), — его химический потенциал, являющийся мерой влияния данного вещества на термодинамическое состояние системы. Уравнение (3) записано при условии постоянства концентраций всех компонентов системы, кроме i-го. Зависимость химического потенциала идеального газа от давления дается формулой

\begin{equation}
 \mu_i = \mu_i^0 + RT \ln{p_i}
\end{equation}
в которой \( \mu_i^0 \) - стандартный химический потенциал i-го компонента при 1 атм.
Если исключительно для простоты записи последующих уравнений представить равновесие (1) в форме

\begin{equation}\label{BA}
 B \Leftrightarrow 2A
\end{equation}
то на основании приведенных выше соотношений условие равновесия реакции \eqref{BA} можно записать в виде

\begin{equation}
 2 \mu_A^0 + RT \ln{p_A^2} - \mu_B^0 - RT \ln{p_B} = 0,
\end{equation}
и после несложного преобразования получим выражение

\begin{equation} \label{RT}
 -RT \ln(p_A^2 / p_B) = 2 \mu_A^0 - \mu_B^0
\end{equation}

в котором комбинация давлений газов в равновесной системе, стоящая в скобках, соответствует константе равновесия

\begin{equation} \label{Koeff}
 K_p = p_A^2 / p_B
\end{equation}
Подставляя \eqref{Koeff} в \eqref{RT}, получим выражение
\begin{equation}
 -RT \ln(K_p) =  2 G_A^0 - G_B^0 = \Delta G^0,
\end{equation}
являющееся условием химического равновесия. В нем скрыта размерность константы равновесия, которая связана с выбором стандартного состояния. Можно записать

\begin{equation}
 \Delta G^0 = -RT ln(K_p / K_p^0)
\end{equation}
причем в стандартном состоянии \(K_p^0\) в скобке равно единице в размерности этого состояния в соответствующей степени. Например, для газов в качестве стандартного состояния в большинстве случаев используют 1 атм, так что в рассматриваемом равновесии \(K_p\) и \(K_p^0\) выражены в атм.
Изменение термодинамического потенциала, в свою очередь, связано с изменениями энтальпии и энтропии в реакции соотношением:

\begin{equation}
 \Delta G^0 = \Delta H^0 - T \Delta S^0
\end{equation}

Величина каждой из составляющих его функций зависит от температуры согласно приближенным уравнениям:

\begin{equation}
 \Delta H_T^0 = \Delta H_{298}^0 + \Delta c_p \Delta T
\end{equation}

\begin{equation}
 \Delta S_T^0 = \Delta S_{298}^0 + \Delta c_p \Delta \ln{T}
\end{equation}
в которых \(\Delta r c_p\) представляет собой разность теплоёмкостей продуктов и исходных веществ, а приращение Т и lnТ отсчитываются от стандартной температуры 298,15 К. Для относительно узких температурных интервалов этими зависимостями можно пренебречь, что приводит к окончательным соотношениям:

\begin{equation}
 \Delta G_T^0 = \Delta H_{298}^0 - T \Delta S_{298}^0
\end{equation}

\begin{equation}
 RT \ln K_p = - \Delta G_T^0 = -\Delta H_{298}^0 + T \Delta S_{298}^0
\end{equation}
которыми предлагается пользоваться в дальнейших расчетах. Более строгую, чем уравнение (15), температурную зависимость константы равновесия можно получить с учетом зависимости \(\Delta rcp(Т)\) и интегрирования уравнений для $\Delta rH_T^0$ и $\Delta rS_T^0$.
	
В практически более удобной записи выражения (8) для константы равновесия реакции (1) в форме (5) используют степень диссоциации \(\alpha\). Предположим, что в замкнутую систему объемом \(V^0\)0 введено \(n^0\) молей газа А, так что суммарная концентрация обоих газов, выраженная в молях А и независящая от Т, составляет

\begin{equation}
 C_0 = n_0 /V_0
\end{equation}

а их полное давление:

\begin{equation}
 p_0 = RT \cdot C_0
\end{equation}

Такое давление имела бы система при температуре Т в условиях полной диссоциации. В иных условиях парциальные давления газов будут определяться уравнением материального баланса:

\begin{equation}
 p_0 = 2p_B + p_A,
\end{equation}

поскольку каждая молекула В содержит две молекулы А.
Если степень диссоциации определить как

\begin{equation}
 \alpha = p_a/p_0 = (RT \cdot C_A)/(RT \cdot C_0)	
\end{equation}

то с учетом (18) получим для давлений газов

\begin{equation}
 p_A = \alpha \cdot p_0,
\end{equation}


\begin{equation}
p_b = (1-\alpha)p_0/2.
\end{equation}

Полное давление равновесной системы будет функцией положения равновесия, т. е. \(0 \leqslant \alpha \leqslant 1\), и составит

\begin{equation}
p = p_A + p_B = \alpha\cdot p_0 + (1- \alpha)p_0/2 = (1+\alpha)p_0/2
\end{equation}

достигая р0 при полной диссоциации и \(p_0/2\)— в ее отсутствие, когда в системе присутствует только В.

Подставляя (20) и (21) в (8), получим выражение для константы равновесия диссоциации:

\begin{equation}
K_p = 2 \alpha^2 p_0(1- \alpha)
\end{equation}

Уравнения (17), (20), (21) и (23) являются основой расчета Kp из экспериментальных данных.



\subsection*{Ход работы}

\begin{figure}[h]
	\begin{center}
		\begin{minipage}{0.45 \linewidth}
			\includegraphics[width = 7cm]{termostat.jpg}
		\end{minipage}
		\qquad
		\begin{minipage}{0.45 \linewidth}
			\includegraphics[width = 7cm]{term.jpg}
		\end{minipage}
	\end{center}
	\caption{Cхема установки}
\end{figure}


\begin{enumerate}
	\item Подготовка установки
	\item Установка температуры
	\item Измерение оптической плотности ячейки
	
\end{enumerate}


\emph{Подготовка установки}

Очистим от загрязнений поверхность ячейки влажной тканью. Откалибруем фотоспектрометр.


\begin{figure}
	\includegraphics[scale=0.8]{section.jpg}
	\caption{установка ячейки}
\end{figure}

\emph{Установка температуры}
Включив термостат, установим на нем начальную температуру (30 $^\circ$C). Дождемся установления равновесного состояния в системе.

\emph{Измерение оптической плотности ячейки}
Проведем измерение оптической плотности ячейки при помощи фотоспектрографа в режиме Spectrum. Повторим пункты б) и в) до температуры 80 $^\circ$C









































\newpage
\section{Комбинации событий}
\subsection{T.1.}
\begin{enumerate}
\item
\[\omega = A \setminus (B \cup C)\]
\item 
\[\omega = (A \cap B) \setminus C \]
\item
\[\omega = A \cap B \cap C\]
\item
\[\omega = A \cup B \cup C\]
\item
\[\omega = A \setminus (B \cup C) \cup B \setminus (A \cup C) \cup
C \setminus (A \cup B) \]
\item
\[\omega = \overline{A \cup B \cup C}\]
\item
\[\omega = \overline{A \cap B \cap C} \]
\end{enumerate}


\newpage
\subsection{T.2.}

\[ \overline{(X \cup A)} \cup \overline{X \cup \overline{A}} = B \]
\[ \overline{(X \cup A) \cap (X \cup \overline{A})} =   \]
\[ = \overline{X \cup (A \cap \overline{A})} = 
\overline{X \cup (\oslash) } = \overline{X}  \]
\[ \Rightarrow X = \overline{B}  \]

\subsection{T.3.}

\begin{enumerate}
\item
\[ \overline{A \cap B} = \overline{A} \cup \overline{B} \]
Пусть произвольное $\omega \in \overline{A \cap B}$
Значит, $\omega$ не принадлежит ни одному из событий одновременно,
т.е.
$\omega \in \overline{A}$ или $\omega \in \overline{B}$

По определению суммы событий:
\[ \omega \in (\overline{A} \cup \overline{B})  \]
Т.е. \[  \overline{A \cap B} \in (\overline{A} \cup \overline{B}) \]

Обратно: пусть произвольное
 $\omega \in (\overline{A} \cup \overline{B})$
 т.е. $ \omega $ не принадлежит не принадлежит событию A и одновременно не принадлежит событию B. Значит по определению произведения событий событие $ \omega $ не принадлежит пересечению.
 По определению разности событий:
 \[ \omega  \in (1 \setminus (A \cap B)) \]
 
 Или
 \[ \omega  \in (\overline{A \cap B)} \]
 Т.е. 
\[\overline{A} \cup \overline{B} \in  \overline{A \cap B} \]

Равенство доказано, т.к. множества его левой и правой частей состоят из одинаковых элементов.
\item 
\[  \]
\item
\[  \]
\item
\[  \]
\item
\[  \]
\item
\[  \]
\item
\[  \]
\end{enumerate}

\newpage






























Наша первая формула $100+100=200$, ага.

\[ 100+100=200 \]

\begin{equation}\label{pifagor}
a^2+b^2=c^2
\end{equation}

Теорему Пифагора \eqref{pifagor} вы знаете с 8 класса\footnote{Определенно знали}. Эта теорема упоминается на странице \pageref{pifagor}.

\subsection{Дроби}

$\frac{1}{3}+\frac{1}{3}=\frac{2}{3}$. Вот вам и дроби\footnote{А это с пятого класса.}. {\scriptsize Так некрасиво.} {\Large Красиво так}:

\[ \frac{1}{3}+\frac{1}{3}=\frac{2}{3} \]


\subsection{Скобки}

\[ (2+3)\cdot 5=25 \]

\[ \left[\frac{4}{2}+3\right]\cdot 5=25 \]

\[ \{2+3\}\cdot 5=25 \]

\subsection{Индексы}

\[ m_1, m_{12}, c^2, c^{22} \]

\subsection{Стандартные функции}

\[ \sin x=0 \]
\[ \arctg x=\sqrt[5]{3} \]
\[ \log_{x-1}{(x^2-3x-4)}\geqslant 2 \]
\[ \lg 10=\ln e \]

\subsection{Функции покрупнее}

$\sum_{i=1}^{n}a_i+b_i$

\[ \sum_{i=1}^{n}a_i+b_i \]

$I=\int r^2dm$

\[I=\int r^2dm \]

\[I=\int_{0}^{1} r^2dm \]

\[I=\int\limits_{0}^{1} r^2dm \]

\subsection{Символы}

\[2\times 2\neq 5 \]

\[x \cap y,  x \cup y\]

\[x\in (-\infty; 0)\]

\[ \triangle ABC = \triangle A_1B_1C_1 \Rightarrow \angle A= \angle A_1\]

\smiley


\newpage
\begin{center}
Вторая часть
\end{center}

\begin{flushright}
Ты будешь доволен собой и женой, своей конституцией куцей!
Но для поэта всемирный запой и мало ему конституций!
\end{flushright}


\begin{itemize}
\item написать скобочный анализ
\item написать Максу
\item 2-я неделя матан
\begin{itemize}
\item задание
\item найти теорию
\end{itemize}
\end{itemize}


\begin{enumerate}
\item написать скобочный анализ
\item написать Максу
\item 2-я неделя матан
\begin{itemize}
\item задание
\item найти теорию
\end{itemize}
\end{enumerate}

\subsection{Диакритические знаки}


\[x\in \mathbb{R} \]






\end{document} % конец документа
